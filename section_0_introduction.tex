\addcontentsline{toc}{chapter}{\introtitle}
\chapter*{\introtitle}
\label{sec:introduction}
\clearpage
\glsresetall

\par Cancer is a heterogeneous disease of common phenotypes but diverse genetic roots. To become malignant, a tumor must overcome several molecular hurdles known as the ``Hallmarks of Cancer'', including sustaining proliferative signalling, evading immune destruction, resisting cell death, and evading growth suppressors, among others \cite{Hallmarks}. The possible genetic mechanisms for achieving these transformations, however, are extremely diverse. Accordingly, the standard of care for cancer treatment varies by tissue, cell of origin, and genetic, epigenetic, and molecular features of the tumor. The practice of personalized oncology, or the tailoring of therapy to the unique combination of molecular features present in a tumor, has become increasingly feasible in recent decades with the advent of genomic data science. Specific molecular targeting of tumor dependency genes has led to improved prognoses of many molecular subtypes of cancer \cite{Vasan_2019}; however, some tumors evade molecular therapy and prognoses for these patients remain poor. 

\par Cancers harboring circular \gls{ecDNA} comprise one such class of tumors for which current therapies often fail. \gls{ecDNA+} tumors are associated with poorer overall survival across many histological tumor types \cite{Kim_2020}, including gliomas \cite{Nikolaev_2014}, neuroblastoma \cite{Koche_2020}, leukemias \cite{Storlazzi_2006}, lung cancer \cite{pongor_2023}, and breast cancer \cite{Vicario_2015}. \textit{In vitro} experiments have shown that \gls{ecDNA} amplification mediates drug resistance in colon cancer \cite{Morales_2009} and glioblastoma \cite{Nathanson_2014} cell lines. Recent continued work on the same glioblastoma cell line showed that \gls{ecDNA} copy number is modulated in response to administration or subsequent withdrawal of targeted therapy, and a similar pattern may be observed in sequential biopsies of \gls{ecDNA+} treatment-na{\"i}ve and post-treatment patient tumors treated with targeted therapies \cite{Lange_2021}. These results constitute strong evidence that \gls{ecDNA} contributes to the evolution of therapy resistance.

\par The mechanistic link between \gls{ecDNA} and poor prognosis is most likely explained by sequence dynamics of \gls{ecDNA} molecules under selective pressure. \gls{ecDNA}s are defined as circular, acentric (ie., lacking a centromere) chromatin bodies hundreds of kilobases to tens of megabases in length \cite{Verhaak_2019}. Tumors containing ecDNA commonly have been observed to accumulate ten- or hundredfold  \gls{cna} per cell of the circularized sequence, driving high oncogenic expression of one or more genes amplified on the ecDNA \cite{Turner_2017}. Because \gls{ecDNA}s lack centromeres, they segregate randomly rather than equally during cell division \cite{Lange_2021}. This non-Mendelian pattern of inheritance allows a tumor to acquire intratumoral copy-number heterogeneity with each cell division, which in turn enables rapid tumor evolution in response to selective pressure \cite{Verhaak_2019,Gu_2020,Lange_2021}.

\par There is accumulating evidence that ecDNA is subject to unique patterns of epigenetic regulation as well. Experiments in glioblastoma and other cell line models have shown that \gls{ecDNA}s have highly accessible chromatin \cite{Wu_2019} and bring functional regulatory elements, or enhancers, in contact with co-amplified oncogene promoters \cite{Morton_2019,Helmsauer_2020}. Furthermore, ecDNA-amplified enhancers have been observed to bind in \textit{trans} to promoters on entirely different molecules, namely chromosomal promoters \cite{Zhu_2021} and other distinct lineages of ecDNA \cite{hung_2021}. Thus, co-amplified regulatory elements on \gls{ecDNA} appear to play previously underappreciated oncogenic roles.
\par New and emerging technologies have powered these recent advances in ecDNA biology. The growth of large cancer genomics cloud data resources such as \gls{pcawg} \cite{pcawg}, St Jude Cloud \cite{stjude}, and \gls{kf} \cite{kidsfirst} have made thousands of cancer genomes available for public research, alongside high-quality clinical metadata. Concurrently, the algorithmic softwares \gls{AA} \cite{AA} and AmpliconClassifier \cite{Kim_2020} have enabled robust and reproducible detection of \gls{ecDNA} from whole genome sequencing. Finally, new technologies facilitate investigation of \gls{ecDNA} sequence and gene transcription: CRISPR-CATCH (ecDNA length and circularity) \cite{crispr-catch}, ATAC-seq (accessible chromatin) \cite{atac-seq}, Hi-C (chromatin interactions) \cite{rao_2014}, single-cell multiome sequencing (single cell resolution gene regulation) \cite{scRNA+ATAC_protocol}, and others. These technologies have enabled investigation of the roles of ecDNA in oncogenesis at ever-greater resolution.

\par While the frequency of ecDNA is known in most adult cancer types and some pediatric tumor types, only isolated case reports describe ecDNA in rare pediatric cancers. Here, The accumulation of \gls{wgs} data in cloud genomic data platforms now offers the opportunity to examine ecDNA in pediatric patient populations at unprecedented resolution. In Chapter~\ref{chap:medullo}, I apply \gls{AA} to a large retrospective cohorts of medulloblastomas to identify and characterize ecDNA across pediatric tumor types. I perform survival analysis with respect to ecDNA and other clinical and genomic metadata to establish association between ecDNA and patient outcomes. 
\par In Chapter~\ref{chap:heterogeneity}, I present new methods to characterize intratumoral heterogeneity of ecDNA in \gls{MB} tumors. Using \gls{FISH}, we label genomic copies of a marker gene on ecDNA in archival \gls{FFPE} tissue of patient tumors, then apply an automated computational pipeline to estimate ecDNA copy number across thousands of cells. This analysis establishes that the ecDNA has greater average copy number and greater variance in copy number per cell than other focal amplifications, consistent with previous reports of copy number heterogeneity in cell lines\cite{Lange_2021}. In addition, we perform high-throughput sequencing of single cells to characterize the transcriptomes of cell types present in ecDNA+ tumors. We show that this approach enables detection and characterization of a distinct \gls{ecDNA+} cell population within a heterogeneous tumor. We anticipate that further application of these approaches may illuminate the evolutionary dynamics of \gls{ecDNA+} tumors under therapeutic pressure.
\par In Chapter ~\ref{chap:epigenomics}, we map the transcriptomes, accessible chromatin, and chromatin interactions of \gls{ecDNA} sequences in patient and model tumors. In doing so, we reveal ectopic chroamatin interactions linking accessible transcriptional enhancer elements to highly-expressed gene loci, both co-amplified on the same ecDNA sequences.  The frequency and diversity of these putative ``enhancer rewiring'' events suggests that dysregulation of gene transcription likely contributes to oncogenic activation of ecDNA-amplified genes.